%!TEX TS-program = xelatex
\documentclass[]{friggeri-cv}
\addbibresource{bibliography.bib}

\begin{document}
\header{Paulo Henrique Lerbach Rodrigues}{}
       {Analista Desenvolvedor Sênior}


% In the aside, each new line forces a line break
\begin{aside}
  \section{contato}
    rua 36 norte
    Águas Claras
    ~
    (61) 8250-4321
    ~
    \href{mailto:palerique@gmail.com}{palerique@gmail.com}
    \href{http://sitedo.ph}{http://sitedo.ph}
    \href{http://github.com/palerique/}{github://palerique}
  \section{idiomas}
    português nativo
    inglês intermediário
  \section{programação}
    {\color{red} $\varheartsuit$} XPTO ALTERAR!!!
    %{\color{red} $\varheartsuit$} JavaScript
    %(ES5, node.js)
    %Python, C, OCaml
    %CSS3 \& HTML5
  \section{Certificações}
    \includegraphics[scale=0.13]{ocpjp.jpg}
    \includegraphics[scale=0.13]{csm.png}
\end{aside}

\section{objetivo}

Atuar em empresas que primem pela adoção de metodologias e princípios ágeis, participando de equipes de desenvolvimento de software que estejam envolvidas em projetos relevantes contribuindo para proporcionar o maior retorno de investimento possível no menor espaço de tempo. 

\section{experiência}

\begin{entrylist}
  \entry
    {desde 01/14}
    {Stefanini}
    {Analista Desenvolvedor Sênior}
    {\emph{Atualmente estou alocado num projeto crítico da Polícia Federal fazendo parte de uma equipe de sêniores especialmente destacados para esse projeto.}}
  \entry
    {2012-2014}
    {4Linux}
    {Consultor Sênior}
    {\emph{Atualmente estou alocado num projeto crítico da Polícia Federal fazendo parte de uma equipe de sêniores especialmente destacados para esse projeto.}}
\end{entrylist}

\section{educação}

\begin{entrylist}
  \entry
    {since 2009}
    {Ph.D. {\normalfont candidate in Computer Science}}
    {DNET/INRIA, LIP/ÉNS de Lyon}
    {\emph{A Quantified Theory of Social Cohesion.}}
  \entry
    {2007–2008}
    {M.Sc. magna cum laude}
    {IXXI, École Normale Supérieure de Lyon}
    {Majoring in Computer Science\\
    Specialization in Complex Systems}
  \entry
    {2006–2007}
    {B.Sc. magna cum laude}
    {École Normale Supérieure de Lyon}
    {Majoring in Computer Science}
  \entry
    {2003–2006}
    {Classes Préparatoires aux Grandes Écoles}
    {Lycée Fénelon, Lycée Louis le Grand, Paris}
    {Preparation for national competitive entrance exams to leading French ``grandes écoles'', specializing in mathematics and physics.}
  \entry
    {2003}
    {French Baccalauréat S. with honors}
    {Lycée Louis le Grand, Paris}
    {Specialization in mathematics and physics}
\end{entrylist}

\section{aplicações}

\begin{entrylist}
  \entry
    {2012}
    {Who did I forget ?}
    {\href{http://whodidiforget.com}{whodidiforget.com}}
    {Guest list recommendation for Facebook events based on friends already attending the event.}
  \entry
    {2011}
    {Fellows}
    {\href{http://fellows-exp.com}{fellows-exp.com}}
    {Automatic community detection among Facebook Friends in order to validate the \emph{cohesion} measure, creation of friend lists.}
  \entry
    {2008}
    {Happy Flu}
    {\href{http://happyflu.com}{happyflu.com}}
    {Experiment aimed to measure viral spreading of content across the blogosphere.}
\end{entrylist}

\section{publicações}


\printbibsection{article}{article in peer-reviewed journal}
\begin{refsection}
  \nocite{*}
  \printbibliography[sorting=chronological, type=inproceedings, title={international peer-reviewed conferences/proceedings}, notkeyword={france}, heading=subbibliography]
\end{refsection}
\begin{refsection}
  \nocite{*}
  \printbibliography[sorting=chronological, type=inproceedings, title={local peer-reviewed conferences/proceedings}, keyword={france}, heading=subbibliography]
\end{refsection}
\printbibsection{misc}{other publications}
\printbibsection{report}{research reports}

\end{document}
