\begin{document}
\header{Paulo Henrique Lerbach Rodrigues}{}
       {Analista Desenvolvedor Sênior}


% In the aside, each new line forces a line break
\begin{aside}
  \section{contato}
    rua 36 norte
    Águas Claras
    ~
    (61) 8250-4321
    ~
    \href{mailto:palerique@gmail.com}{palerique@gmail.com}
    \href{http://sitedo.ph}{http://sitedo.ph}
    \href{http://github.com/palerique/}{github://palerique}
  \section{idiomas}
    português nativo
    inglês intermediário
  %\section{programação}
    %{\color{red} $\varheartsuit$} XPTO ALTERAR!!!
    %{\color{red} $\varheartsuit$} JavaScript
    %(ES5, node.js)
    %Python, C, OCaml
    %CSS3 \& HTML5
  \section{Certificações}
    \includegraphics[scale=0.13]{ocpjp.jpg}
    \includegraphics[scale=0.13]{csm.png}
\end{aside}

\section{objetivo}

Atuar em empresas que primem pela adoção de metodologias e princípios ágeis, participando de equipes de desenvolvimento de software que estejam envolvidas em projetos relevantes contribuindo para proporcionar o maior retorno de investimento possível no menor espaço de tempo. 

\section{experiência}

\begin{entrylist}
  \entry
    {desde 01/14}
    {Stefanini}
    {Analista Desenvolvedor Sênior}
    {\emph{Atualmente estou alocado num projeto crítico da Polícia Federal fazendo parte de uma equipe de seniores especialmente destacados para esse projeto.}}
  \entry
    {2012-2014}
    {4Linux}
    {Consultor Sênior}
    {\emph{Prestei consultoria e suporte à ferramenta de portais eXo Platform e JBoss Portal, tendo atuado em clientes como a Caixa Econômica Federal – CEF e o Ministério de Minas e Energia. Na CEF fomos os responsáveis por diversos de seus portais, dentre eles o que mais tem visibilidade é o Feirão de Imóveis, aquele feirão que a CEF realiza em diversos locais e foi bastante veiculado pela televisão. No MME implementamos o portal principal em parceria com a Red Hat.
Fui o responsável pela implantação de toda a infra de integração contínua utilizada no desenvolvimento pela 4Linux, instalando as ferramentas – jenkins, sonar, gitlab -, integrando-as.
Também fui o mentor e executor da implantação de ambiente semelhante na Cabal Brasil.
Fui o responsável por todo o desenvolvimento do sistema que integra esse ambiente à estrutura utilizada em treinamentos pela 4Linux, bem como da adaptação dos cursos Java para a utilização desse ambiente para a correção automática dos exercícios dos alunos utilizando TDD.
Fui o responsável pela revisão e modernização de todos os cursos da formação Java, incluindo refazer todo o material fornecido aos alunos – slides, apostilas, vídeo aulas - e a curadoria de todo o conteúdo.
Participei do desenvolvimento da ferramenta BPM que é utilizada pelo grupo CDN.}}
  \entry
    {2011-2012}
    {Poliedro - Agência Nacional de Águas - ANA}
    {Analista de Manutenção e Implementação}
    {\emph{Participei do desenvolvimento de sistemas utilizando: Threads; Web Service;  API de persistência Java – JPA; Enterprise JavaBeans – EJB; API Java para Web Services baseados em XML – JAX-WS; servidor Web Logic; Swing; Framework de desenvolvimento de Aplicações ADF da Oracle; JSF.}}
  \entry
    {2011-2013}
    {Bluestar}
    {Professor}
    {\emph{Ministrei cursos de Java para a Web – JavaServer Pages – JSP básico e avançado, abordando temas como Servlets, JavaServer Standard Tag Library – JSTL, outras Taglibs, Expression Language, AJAX, JQuery, JavaScript, HTML Básico, Alguns padrões de arquitetura – MVC, Factory, DAO Factory, Frameworks Struts2 e Hibernate ambos com anotations e XML. Além disso ministro os cursos de Banco de Dados para Programadores – PostgreSQL e Oracle.}}
\end{entrylist}

\newpage

\begin{entrylist}
  \entry
    {2009-2011}
    {Sectio Aurea}
    {Analista Desenvolvedor}
    {\emph{Trabalhei: no desenvolvimento e manutenção de sistemas web; com Orientação a Objeto; com servidores Jboss e Tomcat; com os frameworks: Spring, Struts, Hibernate; com J2EE, EJB, JSP, JSTL; com Bancos de Dados Oracle e PostgreSQL; com PL/SQL; testes unitários com Junit e TestNG; UML; RIA; Flex; ActionScript; Linux; Mac OS; Windows; HTML; CSS; Eclipse; SVN; JavaScript; JQuery e AJAX.}}
\end{entrylist}

\section{formação}

\begin{entrylist}
  \entry
    {2011-2014}
    {Especialização {\normalfont em Engenharia de Software}}
    {Fundação Universa}
    {\emph{Trabalho de Conclusão: Utilização de BDD no Desenvolvimento de Software.}}
  \entry
    {2007-2008}
    {Especialização {\normalfont em Gestão Pública}}
    {Fortium}
    {\emph{Trabalho de Conclusão: Discricionariedade em Concursos.}}
\end{entrylist}

\section{cursos}

\begin{entrylistii}
  \entryii
    {2014 - 20h}
    {JavaScript, a linguagem}
    {Caelum}  
  \entryii
    {2014 - 16h}
    {Grunt: automação de tarefas front-end}
    {Caelum}  
  \entryii
    {2014 - 12h}
    {Banco de dados e SQL}
    {Caelum}  
  \entryii
    {2014 - 20h}
    {Design Patterns para bons programadores}
    {Caelum}  
  \entryii
    {2014 - 16h}
    {AngularJS: o framework MVC da Google}
    {Caelum}
  \entryii
    {2014 - 20h}
    {JPA 2: Introdução à persistência de dados com JPA e Hibernate}
    {Caelum}
  \entryii
    {2013 - 40h}
    {Java EE avançado e Web Services}
    {Caelum}
  \entryii
    {2013 - 20h}
    {JSF 2: simplicidade e produtividade na Web}
    {Caelum}
  \entryii
    {2013 - 12h}
    {Eclipse: produtividade extrema}
    {Caelum}
  \entryii
    {2013 - 12h}
    {Primeiros passos com Java}
    {Caelum}
  \entryii
    {2013 - 16h}
    {Git: Trabalho em Equipe com Controle e Segurança}
    {Caelum}
  \entryii
    {2013 - 12h}
    {Maven: gerenciando dependências}
    {Caelum}
  \entryii
    {2013 - 12h}
    {Testando comportamento através de mocks}
    {Caelum}
  \entryii
    {2013 - 12h}
    {Jenkins: Automação de deploy e Integração Contínua}
    {Caelum}
  \entryii
    {2013 - 12h}
    {Introdução ao cloud do EC2 no Amazon Web Services}
    {Caelum}
  \entryii
    {2010 - 72h}
    {Java e Orientação a Objeto – FJ11 e 19}
    {Caelum}
  \entryii
    {2010 - 245h}
    {Java Profissionalizante}
    {Bluestar}
  \entryii
    {2003 - 72h}
    {Web Developer}
    {EIBsBNET}
\end{entrylistii}

\newpage

\section{qualificações}

Programação Orientada a Objetos (OO, Design Patterns, UML); Principais frameworks (Spring, Hibernate, Struts); Plataforma JEE e Java para Web (EJB 3, JSF, JSP, Servlet, JPA, JDBC, JSTL); Servidores de Aplicação (JBoss, Tomcat, Weblogic); Bancos de Dados (PostgreSQL, Oracle, HSQLDB, H2); Outras Ferramentas (jUnit, TestNG, Log4j, SubVersion, GIT, Redmine, Trac, Eclipse, Intellij IDEA); JavaScript, JQuery, AJAX.

\section{competências}

Consigo aprender sozinho – com bastante facilidade, lendo e pesquisando – novas tecnologias, entendendo bem o funcionamento delas e conseguindo propor soluções baseadas nessas tecnologias. Estou permanentemente em busca de novos conhecimentos, sempre fazendo novos cursos, atento às novidades do mercado, lendo as notícias e o blog de especialistas da área.
Sei preparar um bom curso e treinar pessoas.
Consigo propor soluções criativas e inovadoras.
Lido bem com o stress e gosto muito de trabalhar direcionado a resultados.
Consigo perceber como operacionalizar as estratégias da organização, percebendo-a de forma sistêmica, global.
Consigo motivar pessoas e equipes.
Entendo e abstraio, com muita facilidade, diversos domínios, conseguindo representá-lo de forma simples e ubíqua, para um usuário leigo, ou nas minúcias para um time de desenvolvedores.
Aceito desafios, mesmo os mais complicados.
Gosto de solucionar problemas que outros não conseguiram solucionar.

%\newpage

%\section{aplicações}

%\begin{entrylist}
%  \entry
%    {2012}
%    {Who did I forget ?}
%    {\href{http://whodidiforget.com}{whodidiforget.com}}
%    {Guest list recommendation for Facebook events based on friends already attending the event.}
%  \entry
%    {2011}
%    {Fellows}
%    {\href{http://fellows-exp.com}{fellows-exp.com}}
%    {Automatic community detection among Facebook Friends in order to validate the \emph{cohesion} measure, creation of friend lists.}
%  \entry
%    {2008}
%    {Happy Flu}
%    {\href{http://happyflu.com}{happyflu.com}}
%    {Experiment aimed to measure viral spreading of content across the blogosphere.}
%\end{entrylist}

%\section{publicações}

%\section{publicações2}

\printbibsection{article}{article in peer-reviewed journal}
\begin{refsection}
  \nocite{*}
  \printbibliography[sorting=chronological, type=inproceedings, title={international peer-reviewed conferences/proceedings}, notkeyword={france}, heading=subbibliography]
\end{refsection}
\begin{refsection}
  \nocite{*}
  \printbibliography[sorting=chronological, type=inproceedings, title={local peer-reviewed conferences/proceedings}, keyword={france}, heading=subbibliography]
\end{refsection}
\printbibsection{misc}{other publications}
\printbibsection{report}{research reports}

\end{document}
